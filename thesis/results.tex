\chapter{Results}

\section{Results}

The classification results are tabulated below. The best performing algorithm
random forest was able to classify HTTP over Tor with 93.7\% accuracy and a
false positive rate of 3.7\%. HTTPS over Tor was more easily identified with a
97.7\% accuracy and low false positive rate of 0.3\%.

\begin{table}[H]
  \begin{tabular}{lrrr}
    \toprule
    & True Positive Rate & False Positive Rate & ROC \\
    \midrule
    Random Forest\\
    \midrule
    HTTPS & 0.957 & 0.036 & 0.99\\
    HTTP over Tor & 0.937 & 0.037 & 0.986\\
    HTTPS over Tor & 0.977 & 0.003 & 0.999\\
    Weighted Avg. & 0.954 & 0.03 & 0.99\\
    \midrule
    j4.8 With 10 fold cross validation\\
    \midrule
    HTTPS & 0.951 & 0.04 & 0.989\\
    HTTP over Tor & 0.978 & 0.043 & 0.98\\
    HTTPS over Tor & 0.97 & 0.007 & 0.992\\
    Weighted Avg. & 0.964 & 0.018 & 0.986\\
    \midrule
    Random Tree\\
    \midrule
    HTTPS & 0.961 & 0.046 & 0.963\\
    HTTP over Tor & 0.906 & 0.04 & 0.94\\
    HTTPS over Tor & 0.955 & 0.01 & 0.972\\
    Weighted Avg. & 0.941 & 0.037 & 0.957\\
    \midrule
    Adaboost\\
    \midrule
    HTTPS & 0.95 & 0.001 & 0.975\\
    HTTP over Tor & 0.999 & 0.324 & 0.838\\
    HTTPS over Tor & 0 & 0 & 0.777\\
    Weighted Avg. & 0.785 & 0.109 & 0.891\\
    \bottomrule
  \end{tabular}
  \caption{Results}
  \label{table:results}
\end{table}

\chapter{Discussion of Results}

\section{Limitations}

The classifiers developed within this experiment have been trained against
a relatively small sample set of simulated data, which may only represent
a small fraction of the width and breadth of conditions available on a wider
area network. The characteristics that allow for classification of Tor network
may be masked by the natural variability introduced by a heterogeneous network.

\section{Addressing Research Questions}

\begin{enumerate*}
  \item{Can the traffic be classified using automated matching techniques?}

  Yes, the classification algorithms trialled were all able to identify Tor traffic
  with reasonable accuracy.

  \item{Do these networks have characteristics that make them readily
  distinguishable using heuristics based matching?}

  Insert answer here.

  \item{Do they have characteristics that make them distinguishable using a
  machine learning technique?}

  Insert answer here.

  \item{How long does a user have to be connected to the network before a
  confident match can be made?}

  Insert answer here.

\end{enumerate*}


\section{Possible Implications of Results}

The research results suggest that the strategies used by anonymous networks to
hide the identity of network participants, may make those anonymous networks
more easy to distinguish from networks and software that do not provide any
anonymity.

Research has shown that the low latency design of Tor means it is vulnerable to
traffic analysis attacks that correlate traffic patterns with users. This
vulnerability is indicative of the subtle amount of overhead incurred when
sending traffic through Tor. This should make it difficult to identify whether a
connection is using Tor VS not using Tor, though it is clearly still possible.
Reducing this overhead further may reduce the success of Tor traffic
identification.

Stronger anonymity providing networks utilize more sophisticated techniques to
hide individual identities such as batching, queuing and normalization. While
they may be more resistant to identity uncovering attacks, this may also make
them more easily identifiable by classification techniques.


\chapter{Conclusion}

\section{Conclusion}

Using Tor as a communications proxy incurs some overhead, sufficient enough that
when using a Tor proxy in controlled conditions, Tor traffic can be
distinguished from regularly encrypted traffic. This overhead may be sufficient
enough that Tor nodes in a real world network can be identified by using readily
available eavesdropping techniques.

However, this experiment was based on a small set of simulated data, with which
it would be impossible to cover all the possible real world conditions. The
variability present in the real Tor network may make this classification
technique impossible. Further research needs to be conducted with live packet
traces from real participants in the Tor network.
